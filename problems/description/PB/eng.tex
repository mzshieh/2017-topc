\begin{center}
    {\LARGE Problem B}\vspace{1mm}\\
    {\Large The Combination of Poker Cards}\\
    {Time limit: 1 second}\\
    {Memory limit: 512 megabytes}
\end{center}

\textbf{\large Problem Description}

Poker is a popular card game worldwide, which was played from the 18th century in
a variety of forms. A standard deck of poker cards has 52 cards divided into four suits,
with each suit having the 13 ranks. In the Chinese area, ``Big Two'' and ``Thirteen
Cards'' are two popular poker games based on the ranking of different combinations
of cards. If we want to design such computer poker games, we need to write a
program that can recognize card combinations. Suppose that we simplify the
requirement as follows: Without considering the types of suits but only concerning
about the ranks of cards, try to determine the card combination from any six given
cards. If we use integer numbers to represent card ranks, the possible card
combinations are described below:
\begin{itemize}
\item single: six different numbers, e.g., 2 5 7 10 9 8
\item one pair: one pair of equal numbers, e.g., 4 4 7 10 8 9
\item two pairs: two pairs of equal numbers, e.g., 8 8 3 3 6 7
\item three pairs: three pairs of equal numbers, e.g., 8 8 3 3 7 7
\item one triple: three equal numbers, e.g., 2 2 2 7 5 6
\item two triples: two suits of three equal numbers, e.g., 2 2 2 7 7 7
\item tiki: four equal numbers, e.g., 5 5 5 5 9 8
\item tiki pair: four equal numbers and another one pair, e.g., 5 5 5 5 9 9
\item full house: three equal numbers and another one pair, e.g., 3 3 3 9 9 7
%\item straight: five sequential numbers, e.g., 4 5 6 7 8 10
\end{itemize}
Suppose that we use integer numbers from 1 to 13 to represent the card ranks. 
Please write a
program to determine the card combination from six input numbers.

\textbf{\large Input Format}

The first input line contains one integer number $T$, indicating the number of test
cases. Each test case includes six integer numbers (with each number from 1 to 13)
that are separated by a single white space.

You may assume:
\begin{itemize}
    \tightlist{}
    \item $1 \le T \le 25$
    \item In every test case, any number appears at most 4 times.
\end{itemize}


\textbf{\large Output Format}

Show the name of the card combination for each input test case using the
namespace of single, one pair, two pairs, three pairs, one triple, two triples, tiki, tiki
pair, and full house.

\textbf{\large Sample Input}

\begin{verbatim}
5
4 3 4 3 12 10
5 4 2 3 6 12
10 12 10 12 12 8
8 5 8 8 5 5
2 10 6 10 10 10
\end{verbatim}

\textbf{\large Sample Output}
\begin{verbatim}
two pairs
single
full house
two triples
tiki
\end{verbatim}
