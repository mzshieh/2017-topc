\begin{center}
{\LARGE 2017 ACM-ICPC Asia Taiwan Online Programming Contest}\\
{\LARGE 競賽資訊}
\end{center}

%\section*{競技倫理}
%參賽者應尊重對手,遵守規則,爭取個人、隊伍、學校的榮耀,維護社群的名譽。

\section*{競賽規則}
違反下列規則,將導致參賽者失去參賽資格。
\begin{enumerate}
\item[一、]不得使用任何機器可讀的資料,如預先寫好存放於電腦中的程式碼。但可以使用紙本資料,如教科書、字典、筆記以及列印好的紙本程式碼。
\item[二、]在比賽過程中,參賽者只能與隊友討論。競賽期間與教練或其他隊伍聯繫均屬違規行為。
\item[三、]參賽者只能夠透過網路下載題目敘述、上傳解答程式碼、提問澄清疑點與查看計分板。使用網路存取其他資訊均屬違規。
\item[四、]每個隊伍僅可使用一台電腦撰寫程式與上傳程式碼。於競賽期間除使用印表機列印題目與程式碼以及透過額外的螢幕閱讀題目之外,不得使用任何其他電子裝置。
\item[五、]不得做出任何意圖妨礙比賽進行及影響比賽公平性的惡意行為。
\end{enumerate}

\section*{計分與排名}
\begin{enumerate}
\item[一、]違反競賽規則以致失去參賽資格者,不予計分與排名。
\item[二、]本次競賽僅提供 C、C++、Java、Python ,不接受其他程式語言。
程式需能正常編譯執行,編譯時可使用的記憶體、時間、輸出上限分別為
3.5 gigabytes、20秒、1000 megabytes。
程式在時間限制內計算完畢輸出正確答案,才算答對。以下為裁判系統常見回應:
{	\setlength{\parskip}{1pt}
	\begin{itemize}
    \item Compilation Error: 語法有錯誤或其他因素以致於無法編譯或執行。
    \item Time-Limit Exceeded: 未能在時間限制內執行完畢。
    \item Run-Time Error: 執行時錯誤,程式無法正常執行完畢。
    \item Wrong Answer: 在時限內執行完畢但輸出錯誤。
    \item Yes: 答對。
    \end{itemize}
}
\item[三、]隊伍以解題數量多者排名較前,解題數量相同時,以總消耗時間少者排名較前。答對的題目的消耗時間計算方式為比賽開始至解出題目所消耗的分鐘數。如解出前有答錯,每答錯一次需要另加 20 分鐘。總消耗時間為所有答對題目的消耗時間加總。未答對的題目不計消耗時間。
\item[四、]如果在前述計分規則任兩隊無法分出勝負時,則依據兩隊每題第一份答對的
程式碼之 Run ID 決勝: Run ID 最大值較大者負。
\end{enumerate}

\newpage
