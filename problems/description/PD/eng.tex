\begin{center}
    {\LARGE Problem D}\vspace{1mm}\\
    {\Large Mixing Coins}\\
    {Time limit: 5 seconds}\\
    {Memory limit: 512 megabytes}
\end{center}

\textbf{\large Problem Description}

Misaka likes to shoot coins as a powerful railgun.

She prepares a line of coins to fight crime. To produce a stronger coin, she
mixes coins together. However, coins with different materials are not compatible
to each other, so she only mixes coins with same material together.

Here's the steps Misaka makes coins:

\begin{enumerate}
	\item Find first three consecutive coins with same material from the
	beginning of the line
	\item Take them out from the line
	\item Mix together and produce a new coin with same material
	\item Put the new coin at the end of line
\end{enumerate}

She repeatedly do these steps until she can't produce new coins anymore.

Misaka wants to know how many coins she will have. Please help her count coins
rapidly!

\textbf{\large Input Format}

On the first line there is a single integer $T$ indicating the number of test
cases.

The first line of each test case contains an integer $N$ indicating the number
of groups of consecutive coins Misaka has. All coins are in a single line.

Then $N$ lines follow, each line containing a character $c_i$ and an integer
$n_i$, denoting that there are $n_i$ consecutive coins with material $c_i$ for
$i$-th group of consecutive coins, behind $(i-1)$-th.

You may assume:
\begin{itemize}
    \tightlist{}
    \item $T \leq 10$
    \item $1 \leq N \leq 10^5$
    \item $1 \leq n_i \leq 10^9$
    \item $c_i$ is an uppercase alphabet, $c_i \neq c_{i+1}$ for $1 \leq i < N$
\end{itemize}

\textbf{\large Output Format}

For each test case, output an integer in one line, indicating the number of
coins after Misaka doing the steps of making coins as many as possible.

\textbf{\large Sample Input}

\begin{verbatim}
2
3
A 3
B 1
A 2
3
A 2
B 3
A 2
\end{verbatim}

\textbf{\large Sample Output}

\begin{verbatim}
2
3
\end{verbatim}
