\begin{center}
    {\LARGE Problem E}\vspace{1mm}\\
    {\Large Fences}\\
    {Time limit: 3 seconds}\\
    {Memory limit: 512 megabytes}
\end{center}

\textbf{\large Problem Description}

Your friend, Donald, has a villa surrounded by two tiers of fences, and he 
wants to calculate the area of land between them. He can measure the length 
of any fence, but Donald has no idea on calculating the area. 
Watson, one of Donald's friends, notices that the fences are probably built by 
a computer scientist mastering the knowledge of computational geometry, 
because the following facts are no coincidence.
\begin{itemize}
\item The shape of the land inside the outer tier is a perfect circle $C$. 
Let $B$ denote the set of points on the boundary of $C$.
\item The shape of the land inside the inner tier is a non-self-intersecting polygon $P$ of $n$ vertices. I.e., two edges do not intersect if they don't share a common vertex. Let $V$ denote the set of vertices of $P$. 
\item All vertices of $P$ have identical minimum distances to $C$. 
In other words, for distinct vertices $(x_u,y_u),(x_v,y_v)\in V$, 
we have 
$$\min_{(x,y)\in B}\sqrt{(x-x_u)^2+(y-y_u)^2}=\min_{(x,y)\in B}\sqrt{(x-x_v)^2+(y-y_v)^2}.$$
 
\end{itemize}
Suddenly, you know how to calculate the area of land between the two tiers 
of fences from the total length $c$ of outer tier and the lengths 
$\ell_1,\dots,\ell_n$ of the $n$ edges of $P$. Note that Donald can measure 
these length. Could you help him to calculate the area?

\textbf{\large Input Format}

The first line of the input contains a positive integer $T$ indicating the 
number of test cases. Each test case consists of two lines.
The first line contains two numbers $c$ and $n$ separated by a space. $c$ is 
the total length of the outer tier, i.e., $c$ is the perimeter of $C$. 
$n$ is the number of vertices of P. The second line contains $n$ positive 
integers $\ell_1,\dots,\ell_n$ indicating the lengths of 
edges of $P$. 

You may assume:
\begin{itemize}
    \tightlist{}
    \item $1 \le T \le 100$
    \item $3 \le n \le 10$
    \item $10 \le c\le 1000$
    \item $\ell_1,\dots,\ell_n>0$
    \item $P$ must be inside the circle $C$.
\end{itemize}

\textbf{\large Output Format}

For each case, output the area between the two tiers of fences. 
Your answer will be accepted if the absolute error or the relative error is less than $10^{-6}$.

\textbf{\large Sample Input}

\begin{verbatim}
2
10.0 3
1 1 1
10.0 4
1 1 1 1
\end{verbatim}

\textbf{\large Sample Output}

\begin{verbatim}
7.524734452702549
6.9577471545947684
\end{verbatim}
