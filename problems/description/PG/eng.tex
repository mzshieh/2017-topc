\begin{center}
    {\LARGE Problem G}\vspace{1mm}\\
    {\Large The Jet-Black Wings}\\
    {Time limit: 3 seconds}\\
    {Memory limit: 512 megabytes}
\end{center}

\textbf{\large Problem Description}

``AHHHHHHHHH...''

Eddy, who calls himself "The Jet-Black Wings", is fighting against an evil organization called Dark
Reunion. Then, he startled from the dream.

``I must be more powerful.'' Eddy said to himself in his mind.

Eddy often practice to be a powerful fighter. During his practice, he collects $N$ magic stones. The
$i$-th stone contains $A_i$ units of dark forces. Eddy does the instruction for $Q$ turns, each turn
he has two choices:

\begin{itemize}
\item[$1\ X$:] Use $X$ units of dark forces to all of the magic stones. Thus, the dark forces of the
  $i$-th magic stone changes to $A_i \oplus X$.
\item[$2\ K$:] Sort all the magic stones by their dark forces increasely and sum up the dark forces
  of the first $K$ magic stones.
\end{itemize}

Could you help Eddy to check whether he is correct?

Expression $x \oplus y$ means applying bitwise exclusive or operation to integers $x$ and $y$.
The given operation exists in all modern programming languages, for example, in languages C++ and
Java it is represented as ``\^{}'', in Pascal — as ``xor''.

\textbf{\large Input Format}

On the first line there is a single integer $T$ indicating the number of test cases.

The first line of each test case contains two integers $N$, $Q$, indicating the
number of magic stones and the number of instructions.

The second line of each test case contains $N$ integers $A_1, A_2, \ldots, A_N$, indicating the dark
forces of the $i$-th magic stone.

For the following $Q$ lines, each line contains an instruction ``$1\ X$'' or ``$2\ K$''.

You may assume:
\begin{itemize}
    \tightlist{}
    \item $T \le 1000$
    \item $1 \le N, Q \le 100000$
    \item $0 \le A_i, X < 2^{31}$
    \item $1 \le K \le N$
    \item There are at most $5$ test cases with $N + Q > 200$.
\end{itemize}


\textbf{\large Output Format}

For each ``$2\ K$'' instruction, sum up the dark forces of the first $K$ magic stones after sorted
and output in one line.

\textbf{\large Sample Input}

\begin{verbatim}
1
3 6
4 8 3
1 3
1 1
2 3
1 2
2 2
2 1
\end{verbatim}

\textbf{\large Sample Output}
\begin{verbatim}
17
7
3
\end{verbatim}

