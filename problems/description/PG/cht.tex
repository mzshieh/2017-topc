\begin{center}
    {\LARGE Problem G}\vspace{1mm}\\
    {\Large The Jet-Black Wings}\\
    {Time limit: 3 seconds}\\
    {Memory limit: 256 megabytes}
\end{center}

\textbf{\large Problem Description}

「呃啊...可惡!...要發狂了嗎!」

艾迪,一個自稱為「漆黑之翼」的人,正在與名為「Dark Reunion」的邪惡組織作戰。
接著他就驚醒了,原來只是一場夢阿...

「我一定要變得更強。」 艾迪在心中激勵自己。

為了成為一名強悍的戰鬥者,艾迪認真的鍛鍊自己。
在他的練習中,他收集了 $N$ 顆魔法石頭,第 $i$ 顆魔法石頭有著 $A_i$ 單位的黑暗力量。
艾迪會進行 $Q$ 次操作,每次操作有以下兩種:

\begin{enumerate}
\item[$1\ X$:] 使用 $X$ 單位的黑暗力量於每顆魔法石頭. 因此,第 $i$ 顆魔法石頭的暗黑力量會變成
  $A_i \oplus X$ 單位。
\item[$2\ K$:] 將所有的魔法石頭按照他們的黑暗力量由小到大排序,並加總前 $K$ 顆魔法石頭的黑暗力量。
\end{enumerate}

你能夠幫助艾迪確認他是否正確嗎?

$x \oplus y$ 表示將 $x$ 與 $y$ 進行互斥或操作。這個操作存在於所有常用的程式語言中,例如:C++ 與 Java
即是使用「\^{}」,而 Pascal 則使用「xor」。

\textbf{\large Input Format}

第一行有一個數字 $T$,表示有 $T$ 組測試資料。

每組測試資料的第一行有兩個數字 $N$, $Q$,表示艾迪蒐集的魔法石頭個數與訓練的操作次數。

每組測試資料的第二行有 $N$ 個數字 $A_1, A_2, \ldots, A_N$,
其中 $A_i$ 表示第 $i$ 顆魔法石頭的黑暗力量。

接著有 $Q$ 行,每行為一個操作「$1\ X$」或「$2\ K$」。

可假設:
\begin{itemize}
    \tightlist{}
    \item $T \le 1000$
    \item $1 \le N, Q \le 100000$
    \item $0 \le A_i, X < 2^{31}$
    \item $1 \le K \le N$
    \item 至多只有 $5$ 組測試資料的 $N + Q > 200$。
\end{itemize}

\textbf{\large Output Format}

對於每個操作「$2\ K$」輸出一個數字於一行,表示排序後前 $K$ 顆魔法石頭的黑暗力量總和。

\textbf{\large Sample Input}

\begin{verbatim}
1
3 6
4 8 3
1 3
1 1
2 3
1 2
2 2
2 1
\end{verbatim}

\textbf{\large Sample Output}
\begin{verbatim}
17
7
3
\end{verbatim}

